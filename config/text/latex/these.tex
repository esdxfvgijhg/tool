\documentclass[a4paper, 92pt, twoside, openright]{report}
[flake8]
ignore=D202,D203,D401,N802,E741,W504
max-line-length=120
exclude=st3/mdpopups/png.py,site/*.py


%\usepackage{polyglossia}
%\setdefaultlanguage[calendar=gregorian,numerals=maghrib]{arabic}
%\setotherlanguage{french}
\usepackage[novoc]{arabluatex}
\newfontfamily\arabicfont[Script=Arabic]{Noto Naskh Arabic} % ,scale=2

\usepackage[top=2.5cm, bottom=2cm, left=3cm, right=2.5cm, headheight=15pt]{geometry}


% First page %%%%%%%%%%%%%%%%%%%%%%%%%%%%%%%%%%%%%%%%%%%%%%%%%%%%%%%%%%%%%%%%%%%%%
\def\titlepage{
	\newpage
	\centering
	\linespread{1}
	\normalsize
	\vbox to \vsize\bgroup\vbox to 9in\bgroup
}
\def\endtitlepage{
	\par
	\kern 0pt
	\egroup
	\vss
	\egroup
	\cleardoublepage
}

\pagestyle{empty}
\makeatletter
	\def\@ecole{école}
	\newcommand{\ecole}[1]{\def\@ecole{#1}}

	\def\@specialite{Spécialité}
	\newcommand{\specialite}[1]{\def\@specialite{#1}}

	\def\@directeur{directeur}
	\newcommand{\directeur}[1]{\def\@directeur{#1}}

	\def\@encadrant{encadrant}
	\newcommand{\encadrant}[1]{\def\@encadrant{#1}}
	
	\def\@jurya{}{}{}
	\newcommand{\jurya}[3]{\def\@jurya{#1,	& #2	& #3\\}}
	
	\def\@juryb{}{}{}
	\newcommand{\juryb}[3]{\def\@juryb{#1,	& #2	& #3\\}}
	
	\def\@juryc{}{}{}
	\newcommand{\juryc}[3]{\def\@juryc{#1,	& #2	& #3\\}}
	
	\def\@juryd{}{}{}
	\newcommand{\juryd}[3]{\def\@juryd{#1,	& #2	& #3\\}}
	
	\def\@jurye{}{}{}
	\newcommand{\jurye}[3]{\def\@jurye{#1,	& #2	& #3\\}}
	
	\def\@juryf{}{}{}
	\newcommand{\juryf}[3]{\def\@juryf{#1,	& #2	& #3\\}}
	
	\def\@juryg{}{}{}
	\newcommand{\juryg}[3]{\def\@juryg{#1,	& #2	& #3\\}}
	
	\def\@juryh{}{}{}
	\newcommand{\juryh}[3]{\def\@juryh{#1,	& #2	& #3\\}}
	
	\def\@juryi{}{}{}
	\newcommand{\juryi}[3]{\def\@juryi{#1,	& #2	& #3\\}}
\makeatother

\newcommand\BackgroundPic{%
	\put(0,0){%
		\parbox[b][\paperheight]{\paperwidth}{%
			\includegraphics[height=0.45\paperheight]{Bordure.png}%
			\vfill
		}
	}
}
\newcommand\EtiquetteThese{%
	\put(0,0){%
		\parbox[t][\paperheight]{\paperwidth}{%
			\hfill
			\colorbox{blue}{
				\begin{minipage}[b]{3em}
					\centering\Huge\textcolor{white}{T\\H\\E\\S\\E\\}
					\vspace{0.2cm}
				\end{minipage}
			}
		}
	}
}

\makeatletter
\newcommand{\firstpagex}{
	\newgeometry{top=2.5cm, bottom=2cm, left=2cm, right=1cm}
	\AddToShipoutPicture*{\BackgroundPic}
	\AddToShipoutPicture*{\EtiquetteThese}
	\begin{titlepage}
		\centering
			\includegraphics[width=0.2\textwidth]{upmc}
			\hfill
			\includegraphics[width=0.2\textwidth]{sorbonne}\\
		\vspace{1cm}
			{\Large \'{E}cole doctorale 364 : Sciences Fondamentales et Appliquées}\\
		\vspace{1cm}
			{\huge {\bfseries Doctorat ParisTech}\\
		\vspace{0.5cm}
			TH\`{E}SE}\\
		\vspace{1cm}
			{\bfseries pour obtenir le grade de docteur délivré par}\\
		\vspace{1cm}
			{\huge\bfseries \@ecole}\\
		\vspace{0.5cm}
			{\Large{\bfseries Spécialité doctorale ``\@specialite''}}\\
		\vspace{2cm}
			\textit{présentée et soutenue publiquement par}\\
		\vspace{0.5cm}
			{\Large {\bfseries \@author}} \\
		\vspace{0.5cm}
			le \@date \\
		\vfill
			{\LARGE \color[rgb]{0,0,1} \bfseries{\@title}} \\
		\vfill
			Directeur de thèse : {\bfseries \@directeur}\\
			Co-encadrant de thèse : {\bfseries \@encadrant}\\
		\vfill
		\begin{tabular}{>{\bfseries}llr}
			\large Jury\\
			\@jurya
			\@juryb
			\@juryc
			\@juryd
			\@jurye
			\@juryf
			\@juryg
			\@juryh
			\@juryi
		\end{tabular}
		\vfill
	\end{titlepage}
	\restoregeometry
}
\makeatother

\renewcommand\firstpage[1]{
	\xinput{#1/00-firstpage}
	\firstpagex
	\preface{#1}
}


% Preface %%%%%%%%%%%%%%%%%%%%%%%%%%%%%%%%%%%%%%%%%%%%%%%%%%%%%%%%%%%%%%%%%%%%%%
\def\abstract{
	% \chapter*{Abstract}
	\begin{center}{
		\large\bf Abstract}
	\end{center}
	\small
	\linespread{1.5}
	\normalsize
}

\def\endabstract{
	\par
}

\newenvironment{acknowledgements}{
	\cleardoublepage
	\begin{center}{
		\large \bf Acknowledgements}
	\end{center}
	\small
	\linespread{1.5}
	\normalsize
}{\cleardoublepage}

\def\endacknowledgements{
	\par
}

\newenvironment{dedication}{
	\small
	\linespread{1.5}
	\normalsize
}{\cleardoublepage}

\def\enddedication{
	\par
}

\usepackage{setspace}
\def\preface#1{
	\pagenumbering{roman}
	\pagestyle{plain}
%	\doublespacing
	\chapter*{Abstract}{}
	\addcontentsline{toc}{chapter}{Abstract}
		\input{#1/01-1-english}

	\chapter*{Résumé}{}
		\input{#1/01-2-french}

	\includepdf[pages=1]{#1/01-3-amazigh.pdf}

	\chapter*{\arb{ملخص}}
%	\addcontentsline{toc}{chapter}{\arb{ملخص}}
		\input{#1/01-4-arabic}

	\chapter*{Acknowledgements}{}
	\addcontentsline{toc}{chapter}{Acknowledgements}
		\input{#1/02-acknowledgements}

	\chapter*{Dedication}{}
	\addcontentsline{toc}{chapter}{Dedication}
	\input{#1/03-dedication}
}

\makeatletter
\def\chapapp2{Chapter}

\def\ps@pheadings{
	\let\@mkboth\markboth
	% modifications
	\def\@oddhead{
		\protect\underline{
			\protect\makebox[\textwidth][l]
			{\sl\rightmark\hfill\rm\thepage/\pageref{pba}}
		}
	}
	\def\@oddfoot{}
	\def\@evenfoot{}
	\def\@evenhead{
		\protect\underline{
			\protect\makebox[\textwidth][l]
			{\rm\thepage\hfill\sl\leftmark}
		}
	}
	% end of modifications
	\def\chaptermark##1{
		\markboth{
			\ifnum \c@secnumdepth >\m@ne
			\chapapp2\ \thechapter. \ \fi ##1
		}{}
	}
	\def\sectionmark##1{
		\markright {
			\ifnum \c@secnumdepth >\z@
			\thesection. \ \fi ##1
		}
	}
}
\makeatother

% Body     %%%%%%%%%%%%%%%%%%%%%%%%%%%%%%%%%%%%%%%%%%%%%%%%%%%%%%%%%%%%%%%%%%%%%%
%uhead
\makeatletter
\def\chapapp2{Chapter}
\def\ps@uheadings{
	\let\@mkboth\markboth
	% modifications
	\def\@oddhead{
		\protect\underline{
			\protect\makebox[\textwidth][l]
			{\sl\rightmark\hfill\rm\thepage/\pageref{pba}}
		}
	}
	\def\@oddfoot{}
	\def\@evenfoot{}
	\def\@evenhead{
		\protect\underline{
			\protect\makebox[\textwidth][l]
			{\rm\thepage/\pageref{pba}\hfill\sl\leftmark}
		}
	}
	% end of modifications
	\def\chaptermark##1{
		\markboth{
			\ifnum \c@secnumdepth >\m@ne
			\chapapp2\ \thechapter. \ \fi ##1
		}{}
	}
	\def\sectionmark##1{
		\markright {
			\ifnum \c@secnumdepth >\z@
			\thesection. \ \fi ##1
		}
	}
}
\makeatother

\makeatletter  %to avoid error messages generated by "\@". Makes Latex treat "@" like a letter
\usepackage{epigraph}
\usepackage{setspace}
\def\body{
%		\cleardoublepage
%		\pagestyle{pheadings}
%		\listoftodos[ToDo]
%		\pagestyle{plain}
		%/
		\cleardoublepage
		\pagestyle{pheadings}
		\dominitoc
		\setcounter{tocdepth}{6}
		\tableofcontents
		\pagestyle{plain}
		%4
		\cleardoublepage
		\pagestyle{uheadings}
		\dominilot
		\listoftables
		\pagestyle{plain}
		%5
		\cleardoublepage
		\pagestyle{uheadings}
		\dominilof
		\listoffigures
		\pagestyle{plain}
		%6
		\cleardoublepage
		\pagestyle{uheadings}
		\chapter*{List of Abbreviations}
		\printacronyms[include-classes=abbrev,name=]
		\addcontentsline{toc}{chapter}{List of Abbreviations}
		\pagestyle{plain}
		%7
		\cleardoublepage
		\pagestyle{uheadings}
		\chapter*{List of Nomenclatures}
		\printacronyms[include-classes=nomencl,name=]
		\addcontentsline{toc}{chapter}{List of Nomenclatures}
		\pagestyle{plain}
		%8
		\cleardoublepage
		\pagestyle{uheadings}
		\chapter*{List of Publications}
		\epigraph{''There's no absolutely reliable way to achieve a great citation. However, hardworking could be fruitful'' -  Eraldo Banovac}{}
		\addcontentsline{toc}{chapter}{List of Publications}
		\input{00/04-publications}
		\pagestyle{plain}

		\cleardoublepage
		\pagestyle{uheadings}
		\pagenumbering{arabic}
		\setcounter{mtc}{9}
		\setcounter{secnumdepth}{6}
		%\setlength{\parskip}{3ex plus 0.5ex minus 0.2ex}
}
\makeatother	%to avoid error messages generated by "\@". Makes Latex treat "@" like a letter

% Appendix %%%%%%%%%%%%%%%%%%%%%%%%%%%%%%%%%%%%%%%%%%%%%%%%%%%%%%%%%%%%%%%%%%%%%%
\let\appendixx\appendix
\def\appendix{
	\oldlabel{pba}
	\appendixx
	\pagestyle{plain}
}


% Table of content %%%%%%%%%%%%%%%%%%%%%%%%%%%%%%%%%%%%%%%%%%%%%%%%%%%%%%%%%%%%%%%%%%%%%
\usepackage[titles]{tocloft}
	\newlistentry[subparagraph]{subsubparagraph}{toc}{5}

	\cftsetindents{part}           {00em}{1em}
	\cftsetindents{chapter}        {00em}{2em}
	\cftsetindents{section}        {00em}{3em}
	\cftsetindents{subsection}     {00em}{4em}
	\cftsetindents{subsubsection}  {00em}{5em}
	\cftsetindents{paragraph}      {05em}{1.5em}
	\cftsetindents{subparagraph}   {05em}{2.5em}
	\cftsetindents{subsubparagraph}{05em}{3.5em}



% Chapter %%%%%%%%%%%%%%%%%%%%%%%%%%%%%%%%%%%%%%%%%%%%%%%%%%%%%%%%%%%%%%%%%%%%%
\usepackage{titlesec, blindtext}
	%\newcounter{subsubparagraph}
	\titleclass{\subsubparagraph}{straight}[\subparagraph]

	\renewcommand{\thesection}{\arabic{section}}
	\renewcommand{\thesubsection}{\arabic{section}.\arabic{subsection}}
	\renewcommand{\thesubsubsection}{\arabic{section}.\arabic{subsection}.\arabic{subsubsection}}

	\renewcommand{\theparagraph}{\Alph{paragraph})}
	\renewcommand{\thesubparagraph}{\Alph{paragraph}.\arabic{subparagraph})}
	\renewcommand{\thesubsubparagraph}{\Alph{paragraph}.\arabic{subparagraph}.\arabic{subsubparagraph})\hspace{10pt}}

	\titleformat{\chapter}[hang]   {\normalfont\Huge \bfseries\color{violet}}  {\thechapter\hspace{20pt}|\hspace{20pt}}  {0pt}  {\Huge\bfseries}
	\titleformat*{\section}        {\normalfont\Large\bfseries\color{violet}}
	\titleformat*{\subsection}     {\normalfont\large\bfseries\color{violet}}
	\titleformat*{\subsubsection}  {\normalfont\large\bfseries\color{violet}}
	\titleformat*{\paragraph}      {\normalfont\large\bfseries\color{violet}}
	\titleformat*{\subparagraph}   {\normalfont\large\bfseries\color{violet}}
	\titleformat{\subsubparagraph} {\normalfont\large\bfseries\color{violet}}  {\thesubsubparagraph}                     {0pt}  {}

	\titlespacing{\section}        {0pt}{1ex plus .5ex minus .5ex} {1ex}
	\titlespacing{\subsection}     {5pt}{1ex plus .5ex minus .5ex} {1ex}
	\titlespacing{\subsubsection}  {10pt}{1ex plus .5ex minus .5ex} {1ex}
	\titlespacing{\paragraph}      {15pt}{1ex plus .5ex minus .5ex} {1ex}
	\titlespacing{\subparagraph}   {20pt}{1ex plus .5ex minus .5ex} {1ex}
	\titlespacing{\subsubparagraph}{25pt}{1ex plus .5ex minus .5ex} {1ex}



% Section %%%%%%%%%%%%%%%%%%%%%%%%%%%%%%%%%%%%%%%%%%%%%%%%%%%%%%%%%%%%%%%%%%%%%
\usepackage[section]{placeins}	% Place un FloatBarrier à chaque nouvelle section
\usepackage[font={small}]{caption}
\usepackage{minitoc}		% Mini table des matières, en français
	\setcounter{minitocdepth}{4}
	\setlength{\mtcindent}{12pt}
	\renewcommand{\mtcfont}{\small\rm}
	\renewcommand{\mtcSfont}{\small\bf}
\usepackage[notbib]{tocbibind}		% Ajoute les Tables	des Matières/Figures/Tableaux à la table des matières




\makeatletter
	\let\stdchapter\chapter
	\renewcommand*\chapter{%
		\removelabelprefix
		\addlabelprefix{\thechapter}
		\@ifstar{\starchapter}{\@dblarg\nostarchapter}}
	\newcommand*\starchapter[1]{
		\stdchapter*{#1}
	}
	\def\nostarchapter[#1]#2{
		\StrCount{#1}{:}[\nbmatch]%
		\StrCut[\nbmatch]{#1}{:}\strchapfirst\strchapsecond%
		\xinput{\strchapfirst*/authors}
		
%		\stdchapter{\thetitle}
		%\epigraph{\strchapsecond}
		\minitoc
		\xinput{#1*/0*}
	}
\makeatother


% Citation %%%%%%%%%%%%%%%%%%%%%%%%%%%%%%%%%%%%%%%%%%%%%%%%%%%%%%%%%%%%%%%%%%%%%
\usepackage{epigraph}
	\setlength\epigraphwidth{8cm}
	\setlength\epigraphrule{0pt}
	\renewcommand\textflush{flushright}
	\renewcommand\epigraphsize{\normalsize\itshape}

\let\oldtitle\title
\renewcommand{\title}[1]{\stdchapter{#1}}

\renewcommand\twocolumn{}

\def\Keywords#1{

}
\def\biblio{
  \printbib
}
